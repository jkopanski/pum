\documentclass[rep.tex]{subfiles}
\begin{document}

\chapter{Zadanie 9}
\label{zad9}
\section{Treść}
Zaprojektować tłumik rezystywny typu~$T$ o tłumieniu~$L = 10~dB$,
który włączony pomiędzy linie długie o impedancjach charakterystycznych $Z_{01} = 50~\Omega$ i $Z_{02} = 60~\Omega$
powinien zapewniać obustronne dopasowanie w nieskończenie szerokim paśmie częstotliwości.
Zaprojektować równoważną wersję tego tłumika typu~$\Pi$.

\section{Rozwiązanie}
W pierwszym kroku należy sprawdzić realizowalność dzielnika.
Należy wyznaczyć stosunek impedancji~$r$:
\begin{equation}
  r = \Big(\frac{Z_{01}}{Z_{02}}\Big)^{\pm1}
\end{equation}
przy czym znak przy wykładniku dobiera się tak, aby:~$r > 1$.

Dla przypadku określonego w treści zadania:
\begin{align}
  r &= \frac{Z_{02}}{Z_{01}} = \frac{60}{50} \nonumber \\
  &= 1.2 \nonumber
\end{align}

Następnie można obliczyć minimalne tłumienie jakie wprowadza dzielnik:
\begin{equation}
  L_{min} = 10 \log(\sqrt{r} + \sqrt{r - 1})
\end{equation}
Podstawiając wartości określone w treści zadania otrzymuję się~$L_{min} = 4.33507363245~dB$ co jest mniejsze od wymaganego~$L = 10~dB$.
Oznacza to, że tłumik jest realizowalny.

Projekt tłumików zaczyna się od przekształcenia wartości tłumienia z miary decybelowej na liniową:
\begin{equation}
  N = 10^{(\frac{L}{10})} = 10
\end{equation}

\subsection{Dzielnik typu $T$}
W celu zaprojektowania tłumika typu~$T$ wyznacza się wartości rezystancji zgodnie ze wzorami:
\begin{align}
  R_3 &= \frac{2\sqrt{N \times Z_{01} \times Z_{02}}}{N - 1} &= 38.490017946~\Omega \\
  R_2 &= Z_{02} \frac{N + 1}{N - 1} - R_3 &= 34.8433153874~\Omega \\
  R_1 &= Z_{01} \frac{N + 1}{N - 1} - R_3 &= 22.6210931651~\Omega
\end{align}

\subsection{Dzielnik typu $\Pi$}
W celu zaprojektowania tłumika typu~$\Pi$ wyznacza się wartości rezystancji zgodnie ze wzorami:
\begin{align}
  R_a &= \frac{(N - 1) \sqrt{Z_{01}Z_{02}}}{2\sqrt{N}} &= 77.9422863406~\Omega \\
  R_b &= \frac{Z_{01}R_a(N - 1)}{R_a(N + 1) - Z_{01}(N - 1)} &= 86.0997286466~\Omega \\
  R_c &= \frac{Z_{02}R_a(N - 1)}{R_a(N + 1) - Z_{02}(N - 1)} &= 132.619585539~\Omega
\end{align}

\end{document}
