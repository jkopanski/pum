\documentclass[rep.tex]{subfiles}
\begin{document}

\chapter{Zadanie 10}
\label{zad10}
\section{Treść}
Zaprojektować schodkowy, ćwierćfalowy transformator impedancji o charakterystyce
równomiernie falistej (Czebyszewa) dopasowujący dwie linie współosiowe o impedancjach
charakterystycznych~$Z_{01} = 30~\Omega$ i $Z_{02} = 75~\Omega$.
Transformator ten powinien zapewniać w paśmie $2 3~GHz$ dopasowanie z $WFS \le 1.12$.
Projekt transformatora wykonać przy założeniu,
że przewody zewnętrzne obu dopasowywanych linii mają średnicę~$a = 7~mm$.
Zaprojektować równoważny wariant tego transformatora w postaci transformatora II klasy,
tj. transformatora złożonego z niewspółmiernych odcinków linii o impedancjach
charakterystycznych~$Z_{01} = 30~\Omega$ i $Z_{02} = 75~\Omega$.

\section{Rozwiązanie}
Projekt transformatora rozpoczyna się od określenia ilości sekcji niezbedznych do realizacji.
Minimalna ilość sekcji potrzebnych do realizacji transformatora jest wieksza lub równa:
\begin{align}
  n &\ge \frac{\operatorname{arch}\Bigg(\frac{R - 1}{\Gamma_d \times (r + 1)}\Bigg)}{\operatorname{arch}\Bigg(]frac{1}{\cos\Big(\pi \frac{1 - \omega/2}{2}\Big)} = 1.47234760626 \\
  n &= 2
\end{align}

\begin{align}
  r &= \frac{Z_{02}}{Z_{01}} &= 2.5 \\
  \Gamma_d &= \frac{WFS - 1}{WFS + 1} &= 0.0566037735849 \\
\end{align}
\end{document}
